\section{ディレクトリ構造}

\subsection{/source}

\texttt{/source} に、\TeX のソースファイル(.tex) を配置します。
\texttt{\textbackslash include}タグで呼ばれるような、
部分ドキュメントもこのディレクトリに配置します。

\subsection{/source/images}

画像ファイルなどをここに配置します。
画像ファイルを文章に挿入するときは、\texttt{images/******} を指定します。

\begin{lstlisting}
\begin{figure}[h]
  \centering
  \includegraphics[width=160pt]{images/latex.pdf}
\end{figure}
\end{lstlisting}

\begin{figure}[h]
  \centering
  \includegraphics[width=160pt]{images/latex.pdf}
\end{figure}

.svgファイルも \texttt{/source/images} に配置することで、
自動でpdf形式の画像に変換してファイルを貼り付けることが可能です(要Inkscape)。

\subsection{/texmf}

このプロジェクトのみに依存している、ローカルな \texttt{texmf} ディレクトリです。
.sty, .cls, .mapファイルを適宜配置することで、プロジェクトのビルド時にロードされます。

\subsection{/build}

ビルドで生成されたファイルがここに格納されます。
